%!TEX encoding = UTF-8 Unicode
\documentclass[onecolumn,12pt]{book}
\usepackage[english,italian]{babel}
\usepackage{inconsolata}
%\renewcommand*\familydefault{\ttdefault} %% Only if the base font of the document is to be typewriter style
\usepackage[T1]{fontenc}

\usepackage{a4wide,Sweave,url}
\usepackage{verbatim}
\usepackage{makeidx}
\usepackage{babelbib}
\usepackage{float}
\usepackage{fancyhdr}
\usepackage[T1]{fontenc}
\usepackage[utf8]{inputenc}
\usepackage{framed}
\usepackage{lipsum}
\usepackage[dvipsnames]{color}
\definecolor{shadecolor}{rgb}{0.9,0.9,0.9}
\usepackage{graphicx}
\usepackage{fancyvrb}
\usepackage{amsmath}

\usepackage{hyperref}
\newenvironment{question}{\item \textbf{Esercizio}\newline}{}
\newenvironment{solution}{\textbf{Soluzione}\newline}{}
\newenvironment{answerlist}{\renewcommand{\labelenumi}{(\alph{enumi})}\begin{enumerate}}{\end{enumerate}}
\definecolor{grigetto}{rgb}{0.9,0.9,0.9}
 
\DefineVerbatimEnvironment{Sinput}{Verbatim} {xleftmargin=2em} \DefineVerbatimEnvironment{Soutput}{Verbatim}{xleftmargin=2em} \DefineVerbatimEnvironment{Scode}{Verbatim}{xleftmargin=2em} \fvset{listparameters={\setlength{\topsep}{0pt}}} \renewenvironment{Schunk}{\small\vspace{\topsep}}{\vspace{\topsep}\normalsize}
\lhead[\thepage]{\today}
%\usepackage{draftwatermark}
\usepackage{wrapfig}
\usepackage{listings}
\pagestyle{fancy}
\newcounter{fnotes}\setcounter{fnotes}{1}
\newcounter{Raction}\setcounter{Raction}{1}
\newcommand{\varia}[1]{\textsl{\textsf{#1}}}
\newcommand{\mytilde}{$\sim$}
\newcommand{\maurizio}[1]{\color{red}#1 \color{black}}
\newcommand{\federico}[1]{\color{green}#1 \color{black}}
 %\newenvironment{question}{\item \textbf{Problema}\newline}{}
%\newenvironment{solution}{\textbf{Soluzione}\newline}{}
%\DefineVerbatimEnvironment{Sinput}{Verbatim} {xleftmargin=2em,
                                            %  frame=single}
\DefineVerbatimEnvironment{Soutput}{Verbatim}{xleftmargin=2em,   frame=single}
 \newenvironment{ese} [1]{\vskip10pt
%\begin{center}
%\begin{minipage}{12cm}
 \markright{\today}
\definecolor{grigetto}{rgb}{0.9,0.9,0.9}
\colorbox{grigetto}{\parbox{\linewidth}{#1}}}
                          {
                         % \end{minipage}
                          %\end{center}
                          \medskip}
 \newcommand{\virgolette}{\selectlanguage{english}\texttt{"}\selectlanguage{italian}}
 \frontmatter\title{Matematica e Statistica con \textsf{R}}
\author{Federico Comoglio e  Maurizio Rinaldi}
\markright{\today}
\lhead{\today}
\renewcommand{\chaptermark}[1]{%
 \markboth{\chaptername
 \ \thechapter.\ #1}{}}
%\renewcommand{\sectionmark}[1]{%
% \markboth{\sectionname
% \ \thesection.\ #1}{}}
\newcommand{\rst}{\textsf{RStudio}~}
\newcommand{\rpr}{\textsf{R}~}
\makeindex
\begin{document}
\Sconcordance{concordance:RbookParte2-knitr.tex:RbookParte2-knitr.Rnw:%
1 67 1 50 0 1 6 10 1 1 4 9 1 1 3 3 1 4 0 5 1 8 0 10 1 17 0 26 1 28 0 6 1 17 %
0 8 1 4 0 6 1 12 0 5 1 7 0 5 1 8 0 7 1 12 0 6 1 12 0 2 1 1 2 3 1 8 0 5 1 4 %
0 3 1 1 3 4 1 7 0 6 1 8 0 7 1 14 0 5 1 14 0 5 1 7 0 12 1 4 0 12 1 8 0 5 1 7 %
0 7 1 17 0 10 1 15 0 5 1 10 0 6 1 10 0 6 1 10 0 7 1 4 0 6 1 4 0 12 1 10 0 8 %
1 24 0 5 1 14 0 8 1 11 0 10 1 10 0 5 1 10 0 6 1 7 0 24 1 31 0 9 1 7 0 9 1 %
17 0 4 1 1 3 3 1 4 0 6 1 8 0 6 1 4 0 5 1 10 0 6 1 16 0 10 1 10 0 4 1 2 3 4 %
1 7 0 8 1 13 0 6 1 20 0 4 1 4 0 7 1 17 0 4 1 8 0 3 1 1 3 4 1 4 0 1 1 1 3 3 %
1 7 0 1 1 1 3 5 1 7 0 7 1 28 0 4 1 28 0 12 1 37 0 8 1 19 0 7 1 25 0 31 1 4 %
0 7 1 7 0 5 1 7 0 4 1 7 0 6 1 7 0 29 1 7 0 4 1 1 2 5 1 10 0 1 1 1 2 3 1 7 0 %
7 1 7 0 18 1 7 0 11 1 7 0 7 1 7 0 6 1 8 0 8 1 8 0 5 1 8 0 11 1 8 0 5 1 7 0 %
4 1 1 9 1 5 4 1 5 0 4 1 5 0 2 2 4 1 22 0 6 1 14 0 5 1 4 0 6 1 2 0 11 1 11 0 %
6 1 9 0 16 1 9 0 4 1 9 0 5 1 9 0 16 1 4 0 19 1 4 0 1 1 1 2 5 1 6 0 6 1 1 2 %
16 1 6 0 16 1 4 0 6 1 2 0 13 1 4 0 6 1 6 0 8 1 7 0 10 1 4 0 8 1 29 0 19 1 4 %
0 4 1 4 0 7 1 6 0 10 1 13 0 10 1 8 0 7 1 13 0 14 1 75 0 6 1 4 0 18 1 4 0 6 %
1 1 2 50 1 1 73 11 1 2 0 32 1 1 14 16 1 1 0 25 1 1 9 7 1 17 0 6 1 4 0 6 1 2 %
0 8 1 6 0 10 1 7 0 6 1 7 0 3 1 1 27 10 1 1 16 19 1 4 0 4 1 7 0 2 1 1 7 4 1 %
1 10 13 1 10 0 22 1 6 0 14 1 6 0 8 1 4 0 4 1 4 0 21 1 17 0 4 1 17 0 5 1 17 %
0 7 1 4 0 5 1 17 0 16 1 4 0 4 1 1 11 17 1 4 0 40 1 17 0 25 1 1 16 20 1 17 0 %
9 1 18 0 10 1 11 0 8 1 11 0 8 1 26 0 9 1 25 0 8 1 4 0 10 1 7 0 7 1 7 0 25 1 %
7 0 16 1 4 0 12 1 7 0 5 1 4 0 5 1 4 0 5 1 4 0 10 1 4 0 9 1 4 0 7 1 4 0 6 1 %
3 0 29 1 12 0 6 1 9 0 11 1 4 0 13 1 37 0 6 1 8 0 6 1 4 0 5 1 2 0 8 1 15 0 %
11 1 4 0 10 1 13 0 7 1 4 0 9 1 4 0 10 1 7 0 7 1 7 0 10 1 6 0 6 1 6 0 10 1 7 %
0 3 1 1 3 6 1 4 0 17 1 13 0 5 1 6 0 6 1 4 0 27 1 4 0 4 1 22 0 11 1 6 0 7 1 %
4 0 11 1 6 0 7 1 23 0 5 1 14 0 10 1 8 0 7 1 8 0 6 1 4 0 8 1 47 0 6 1 11 0 %
11 1 4 0 6 1 6 0 9 1 9 0 2 1 5 0 3 1 5 0 6 1 4 0 13 1 15 0 25 1 4 0 8 1 7 0 %
10 1 35 0 16 1 4 0 26 1 29 0 7 1 6 0 6 1 1 0 12 1 12 0 13 1 4 0 19 1 1 4 4 %
1 4 0 7 1 24 0 7 1 3 0 9 1 7 0 4 1 8 0 37 1 4 0 14 1 2 0 17 1 4 0 7 1 1 0 8 %
1 25 0 5 1 6 0 17 1 6 0 10 1 6 0 7 1 1 2 20 1 6 0 2 1 1 2 8 1 6 0 14 1 6 0 %
21 1 4 0 6 1 1 0 19 1 4 0 16 1 4 0 8 1 6 0 15 1 6 0 13 1 12 0 11 1 6 0 20 1 %
6 0 23 1 4 0 7 1 1 16 31 1 6 0 11 1 4 0 3 1 5 0 3 1 4 0 27 1 4 0 14 1 4 0 %
20 1 4 0 24 1 4 0 16 1 4 0 9 1 4 0 13 1 6 0 22 1 6 0 16 1 4 0 23 1 4 0 18 1 %
8 0 16 1 8 0 18 1 8 0 18 1 4 0 8 1 4 0 7 1 4 0 5 1 4 0 10 1 4 0 10 1 1 2 10 %
1 6 0 9 1 4 0 9 1 4 0 17 1 6 0 30 1 6 0 11 1 4 0 8 1 6 0 18 1 4 0 16 1 4 0 %
16 1 4 0 23 1 4 0 7 1 1 2 3 1 54 0 8 1 56 0 9 1 55 0 9 1 54 0 9 1 55 0 35 1}

\setkeys{Gin}{width=0.7\textwidth}

\begin{Schunk}
\begin{Sinput}
> library(knitr)
> options(formatR.arrow=TRUE,width=50)
> opts_chunk$set(fig.path='figure/graphics-', cache.path='cache/graphics-', fig.align='center', dev='tikz', fig.width=5, fig.height=5, fig.show='hold', cache=TRUE, par=TRUE)
> knit_hooks$set(par=function(before, options, envir){
+ if (before && options$fig.show!='none') par(mar=c(4,4,.1,.1),cex.lab=.95,cex.axis=.9,mgp=c(2,.7,0),tcl=-.3)
+ }, crop=hook_pdfcrop)
\end{Sinput}
\end{Schunk}

\markright{\today}
\thispagestyle{empty}
\maketitle
\newpage
\thispagestyle{empty}
\tableofcontents
\newpage
\thispagestyle{empty}
 \mainmatter
\chapter{Introduzione}
\begin{Schunk}
\begin{Sinput}
> knit_hooks$set(pars = function(before, options, envir) {
+   if (before) graphics::par(options$pars)
+ })
\end{Sinput}
\end{Schunk}

